% ---
% RESUMOS
% ---

% RESUMO em português
\setlength{\absparsep}{18pt} % ajusta o espaçamento dos parágrafos do resumo
\begin{resumo}
 O avanço da tecnologia e o crescimento exponencial da geração de dados têm impulsionado, ano após ano, o uso de algoritmos de aprendizado de máquina em diferentes áreas do conhecimento. Entretanto, muitos desses algoritmos são considerados “caixas-pretas”, devido à complexidade que dificulta a compreensão de como produzem seus resultados. Nesse contexto, surge a área da inteligência artificial explicável (XAI), cujo objetivo é tornar mais transparentes os modelos complexos, fornecendo interpretações sobre seus resultados. Um desafio frequente em análises com múltiplas variáveis é a colinearidade, situação em que uma ou mais variáveis independentes utilizadas no treinamento do modelo apresentam elevada correlação entre si, podendo ser expressas como funções lineares umas das outras. Esse fenômeno pode comprometer a confiabilidade das explicações fornecidas pelas técnicas de XAI. Este trabalho tem o objetivo avaliar o impacto da colinearidade sobre técnicas de inteligência artificial explicável, buscando compreender de que forma essa condição influencia as explicações obtidas.


 \textbf{Palavras-chaves}: inteligência artificial explicável, aprendizado de máquina, colinearidade, interpretação de modelos, transparência.
\end{resumo}

% ABSTRACT in english
\begin{resumo}[Abstract]
 \begin{otherlanguage*}{english}
   The advancement of technology and the exponential growth of data generation have driven, year after year, the use of machine learning algorithms in different fields of knowledge. However, many of these algorithms are considered “black boxes” due to their complexity, which makes it difficult to understand how they produce their results. In this context, the field of explainable artificial intelligence (XAI) has emerged, aiming to make complex models more transparent by providing interpretations of their outcomes. A common challenge in analyses with multiple variables is collinearity, a situation in which one or more independent variables used to train the model are highly correlated with each other and can be expressed as linear functions of one another. This phenomenon can compromise the reliability of the explanations provided by XAI techniques. This work aims to assess the impact of collinearity on explainable artificial intelligence techniques, seeking to understand how this condition influences the obtained explanations.

   \vspace{\onelineskip}
 
   \noindent 
   \textbf{Keywords}: explainable artificial intelligence, machine learning, collinearity, model interpretation, transparency.
 \end{otherlanguage*}
\end{resumo}